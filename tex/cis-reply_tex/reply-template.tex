\documentclass[a4paper,10pt]{jsarticle}

\usepackage{graphicx}

\makeatletter
%%% maketitle再定義
\renewcommand\maketitle[0]{
    \hspace{-1em}
    \fontsize{16}{32}{\textgt{\@title}} \large \\
    \vspace{-1.5em}
    \begin{flushright}
        \@author
    \end{flushright}
}
%%% section再定義
\renewcommand{\section}{
  \@startsection{section}
    {1} % 深さ(sectionが1, subsectionは2等)
    {.5em} % 左インデント量
    {\Cvs \@plus.5\Cdp \@minus.2\Cdp} % 前アキ 見出し上のスペース
    {.5\Cvs \@plus.3\Cdp} % 後アキ 見出し下のスペース 負にすると見出し後のスペース
    {\normalfont\large\headfont\raggedright} % 見出しのフォント
}
\makeatother

\begin{document}

% ------------------------------------------------------
% title
\title{卒業研究中間報告会の指摘事項に対する回答書}
\author{情報工学コース S163043 鈴木健太(平川研究室)}
\maketitle

% ------------------------------------------------------------------------
% ここから本文スタート
% ------------------------------------------------------------------------

\section{なぜ接地を選んだのか.}
足の動きの解析を行い新しい操作系を作りたいというのが最初のモチベーションにあった.そして,周期性を持つ動作なら利用しやすいと考え歩行周期の初期相である初期接地を選んだ.
\section{この研究が何の役に立つか}
歩行周期を推定することにより,足の動きを補助するリハビリテーション分野での活用が考えられる.また,PoseNetによる姿勢推定は複数人を同時に行えるのでセンサーデータでは難しかった多人数の歩容解析が行える可能性がある.
\section{機械学習を用いるのが自然}
以下の2点から今回の研究では機械学習を用いなかった.
\begin{itemize}
  \item 教師あり学習では,教師ラベル付けが必要となり解析処理において時間がかかってしまう.
  \item 教師なし学習では学習データ数を十分に用意する必要がある.
\end{itemize}
\section{足のかかとが地面に着いたときが接地か.}
今回の研究では,ランチョ・ロス・アミーゴ方式\cite{cite1}から足のかかとが地面に着いたときを接地とする.
\section{データが歩行と合っていないなら難しい.}
今回の研究において解析を行った結果,歩行周期と各特徴点データに相関が見られることがわかった.
\section{接地判定ができているか}
今回の研究では,接地判定は出来ていない.
\section{着地点は見えているか}
今回の研究では,特徴点座標を利用しているが着地点の座標の取得,推定は行っていない.
\section{PoseNetは姿勢推定ライブラリの中では最新か.}
PoseNetは2018年3月に公開された機械学習フレームワークのTensorFlow.jsのライブラリの一つである.PoseNet公開後にもさまざまな姿勢推定ライブラリが公開されているので最新ではない.
% ------------------------------------------------------------------------
% 本文ここまで
% ------------------------------------------------------------------------
\begin{thebibliography}{3}
  \bibitem{cite1} Götz-Neumann K:観察による歩行分析.月城慶一,他(訳),医
学書院,東京,2005,pp. 5‒80.
\end{thebibliography}
\end{document}
