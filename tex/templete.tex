\documentclass[a4j,10.5pt]{jreport}
\usepackage[dvipdfmx]{graphicx}
\usepackage{amssymb}
\usepackage{amsmath}
\usepackage{float}


% \usepackage{tabularx}
% \usepackage{multirow}
% \usepackage{slashbox} default commented out
\usepackage{cite}
% \usepackage{supertabular}
% \usepackage{jtygm} default commented out

\makeatletter %% プリアンブルで定義する場合は必須

% \setcounter{secnumdepth}{5}

%\newcommand{\figcaption}[1]{\def\@captype{figure}\caption{#1}}
%\newcommand{\tblcaption}[1]{\def\@captype{table}\caption{#1}}



%---------------------------------------------------------------------
%\setcounter{topnumber}{5}%    ページ上部の図表は 5 個まで
%\def\topfraction{1.00}%       ページの上 1.00 まで図表で占めて可
%\setcounter{bottomnumber}{5}% ページ下部の図表は 5 個まで
%\def\bottomfraction{1.00}%    ページの下 1.00 まで図表で占めて可
%\setcounter{totalnumber}{10}% ページあたりの図表は 10 個まで
\def\textfraction{0.2}%        ページのうち本文が占める割合の下限
%        これを 0 にすると本文が 1 行だけのページが出来る
%        0.04 くらいにするとS 1 行だけのページは防げる
%        0.1 くらいが良いかも知れない
\def\floatpagefraction{0.8}%   図表だけのページは少ないかも
                           %   これだけを図表が占める
%---------------------------------------------------------------------

\renewcommand{\bibname}{参考文献}

\makeatother %% プリアンブルで定義する場合は必須

\begin{document}

\begin{titlepage}

\vspace{40mm}
\begin{center}
{\Large 平成31年度\\卒業論文}\\[80mm]
\end{center}

\begin{center}
{\huge 題名}\\[100mm]
\end{center}

\begin{flushright}
{\large 学籍番号 名前}\\[1mm]
{\large 平成〇年度入学}\\[1mm]
{\large 島根大学 総合理工学部 数理・情報システム学科}\\[1mm]
{\large 情報システムコース}\\[8mm]
{\large 主指導教員: 平川 正人 教授}\\[3mm]
{\large 提出日:令和〇年〇月〇日}\\%
\end{flushright}
\end{titlepage}

\newpage
% \pagenumbering{roman}

\chapter*{概要}
\addcontentsline{toc}{chapter}{概要}

\pagenumbering{roman}

\tableofcontents

\listoftables %表目次

\listoffigures %図目次

\baselineskip = 8mm

\clearpage

\pagenumbering{arabic}

% 英語の概要が必要な場合
% \section*{Abstract}

% 必要に応じてchapter,sectionを追加・変更する.

\chapter{はじめに}
\section{研究背景}

\section{研究目的}

\section{本論文の構成}

\chapter{関連研究}
\section{○○に関する研究}
\section{△△に関する研究}

参考\cite{cite1}.

\chapter{提案手法(システム)}
\section{セクション1}
\begin{itemize}
 \item 箇条書きサンプル その1.
 \item 箇条書きサンプル その2.
\end{itemize}

\section{セクション2}
\begin{enumerate}
    \item サンプル
    \item サンプル
\end{enumerate}

\subsection{subsection}

\chapter{システム評価、実験}

\chapter{実験結果、評価、分析等}

\chapter{終わりに、まとめ等}

\chapter*{謝辞}
\addcontentsline{toc}{chapter}{謝辞}
ここに研究の謝辞.主にご協力いただいた方など.


% \bibliographystyle{jplain}
\begin{thebibliography}{99}
 \bibitem{cite1}http://www.latex-cmd.com/
 \bibitem{cite2}
 \bibitem{cite3}
 \bibitem{cite4}
\end{thebibliography}
\addcontentsline{toc}{chapter}{\bibname}

\chapter*{付録}
\addcontentsline{toc}{chapter}{付録}

\end{document}
